\documentclass[11pt]{article}

\usepackage{float}
\usepackage{hyperref}
\usepackage{graphicx}
% formatting
\usepackage{verbatim}
\usepackage{moreverb}
\usepackage{minted}
\usepackage{parskip}
\usepackage{amsmath}
\usepackage[listings]{tcolorbox}
\usepackage{enumerate}
\let\verbatiminput=\verbatimtabinput
\def\verbatimtabsize{4\relax}

\newcommand{\RepoRootPath}{fpga\_labs\_fa19}

\tcbset{
texexp/.style={colframe=black, colback=lightgray!15,
         coltitle=white,
         fonttitle=\small\sffamily\bfseries, fontupper=\small, fontlower=\small},
     example/.style 2 args={texexp,
title={Question \thetcbcounter: #1},label={#2}},
}

\newtcolorbox{texexp}[1]{texexp}
\newtcolorbox[auto counter]{texexptitled}[3][]{%
example={#2}{#3},#1}

\setlength{\topmargin}{-0.5in}
\setlength{\textheight}{9in}
\setlength{\oddsidemargin}{0in}
\setlength{\evensidemargin}{0in}
\setlength{\textwidth}{6.5in}

% Useful macros

\newcommand{\note}[1]{{\bf [ NOTE: #1 ]}}
\newcommand{\fixme}[1]{{\bf [ FIXME: #1 ]}}
\newcommand{\wunits}[2]{\mbox{#1\,#2}}
\newcommand{\um}{\mbox{$\mu$m}}
\newcommand{\xum}[1]{\wunits{#1}{\um}}
\newcommand{\by}[2]{\mbox{#1$\times$#2}}
\newcommand{\byby}[3]{\mbox{#1$\times$#2$\times$#3}}


\newenvironment{tightlist}
{\begin{itemize}
 \setlength{\parsep}{0pt}
 \setlength{\itemsep}{-2pt}}
{\end{itemize}}

\newenvironment{titledtightlist}[1]
{\noindent
 ~~\textbf{#1}
 \begin{itemize}
 \setlength{\parsep}{0pt}
 \setlength{\itemsep}{-2pt}}
{\end{itemize}}

% Change spacing before and after section headers

\makeatletter
\renewcommand{\section}
{\@startsection {section}{1}{0pt}
 {-2ex}
 {1ex}
 {\bfseries\Large}}
\makeatother

\makeatletter
\renewcommand{\subsection}
{\@startsection {subsection}{1}{0pt}
 {-1ex}
 {0.5ex}
 {\bfseries\normalsize}}
\makeatother

% Reduce likelihood of a single line at the top/bottom of page

\clubpenalty=2000
\widowpenalty=2000

% Other commands and parameters

\pagestyle{myheadings}
\setlength{\parindent}{0in}
\setlength{\parskip}{10pt}

% Commands for register format figures.

\newcommand{\instbit}[1]{\mbox{\scriptsize #1}}
\newcommand{\instbitrange}[2]{\instbit{#1} \hfill \instbit{#2}}



\begin{document}

\def\PYZsq{\textquotesingle}
\title{\vspace{-0.4in}\Large \bf EECS 151/251A FPGA Lab 4:\\Memories, FSMs, Music Streamer\vspace{-0.1in}}

\author{Prof. Borivoje Nikolic and Prof. Sophia Shao \\
TAs: Cem Yalcin, Rebekah Zhao, Ryan Kaveh, Vighnesh Iyer \\ Department of Electrical Engineering and Computer Sciences\\
College of Engineering, University of California, Berkeley}
\date{}
\maketitle


\newcommand{\headertext}{EECS 151/251A FPGA Lab 4: Simulation and Sequential Circuits}
\markboth{\headertext}{\headertext}
\thispagestyle{empty}

\section{Before You Start This Lab}

Before you proceed with the contents of this lab, we suggest that you look through these two documents that will help you better understand some Verilog constructs.

\begin{enumerate}
  \item \href{http://inst.eecs.berkeley.edu/~eecs151/fa19/files/verilog/wire_vs_reg.pdf}{wire\_vs\_reg.pdf} - The differences between wire and reg nets and when to use each of them.
  \item \href{http://inst.eecs.berkeley.edu/~eecs151/fa19/files/verilog/always_at_blocks.pdf}{always\_at\_blocks.pdf} - Understanding the differences between the two types of always @ blocks and what they synthesize to.
\end{enumerate}

\section{Introduction to Inferred Asynchronous ROMs}
An asynchronous memory is a memory block that isn't governed by a clock. In this lab, we will use a Python script to generate a ROM block in Verilog.

A ROM is a read-only memory. This data can be accessed by supplying an address to the ROM; after some time, the ROM will output the data stored at that address. A memory block in general can contain as many addresses in which to store data as you desire. Every address should contain the same amount of data (bits). The number of addresses is called the \textbf{depth} of the memory, while the number of bits stored per address is called the \textbf{width} of the memory.

The synthesizer takes the Verilog you write and converts it into a low-level netlist of the structures are actually used on the FPGA. Our Verilog \textbf{describes} the functionality of some digital circuit and the synthesizer \textbf{infers} what primitives implement the functional description. In this section, we will examine the Verilog that allows the synthesizer to infer a ROM. This is a minimal example of a ROM in Verilog: (depth of 8 entries/addresses, width of 12 bits)

\begin{minted}[tabsize=2]{verilog}
module rom (input [2:0] address, output reg [11:0] data);
  always @(*) begin
    case(address)
      3'd0: data = 12'h000;
      3'd1: data = 12'hFFF;
      3'd2: data = 12'hACD;
      3'd3: data = 12'h122;
      3'd4: data = 12'h347;
      3'd5: data = 12'h93A;
      3'd6: data = 12'h0AF;
      3'd7: data = 12'hC2B;
    endcase
  end
endmodule
\end{minted}

To power our \verb|tone_generator|, we will be using a ROM that is X entries/addresses deep and 24 bits wide. The ROM will contain tones that the \verb|tone_generator| will play. You can choose the depth of your ROM based on the length of the sequence of tones you want to play.

We've provided you with a few scripts that can generate a ROM from either a file with it's contents or even from sheet music. Run these commands from \verb|lab3/|.

\begin{minted}{bash}
python scripts/musicxml_parser.py musicxml/Twinkle_Twinkle_Little_Star.mxl music.txt
python scripts/rom_generator.py music.txt ./lab3.srcs/sources_1/new/rom.v 1024 24
\end{minted}

The first script will parse a MusicXML file and turn it into a list of \verb|tone_switch_periods| for each of the notes for a piece of sheet music. The second script will take that list and turn it into a ROM that's 1024 entries deep with a width of 24 bits.

Take a look at \verb|music.txt| and \verb|src/rom.v|. You can download your own music in MusicXML format from here (\url{https://musescore.org/}) and run it through the same parser; it should ideally only have one part to work properly. You can also directly edit the \verb|music.txt| file to customize the contents of the ROM as you wish.

\section{Design of the music\_streamer}
Open up the \verb|music_streamer.v| file. You will need to modify this module to contain an instance of the ROM you created earlier and logic to address the ROM sequentially to play notes. The \verb|music_streamer| will play each note in the ROM for a predefined amount of time by sending it to the \verb|tone_generator|.

We will play each note for 1/25th of a second. Calculate what that is in terms of 125Mhz clock cycles.

Now let's begin the design of the \verb|music_streamer| itself. Instantiate your ROM in the \verb|music_streamer| and connect the ROM's \verb|address| and \verb|data| ports to wire or reg nets that you create in your module.  The \verb|last_address| port outputs the last address in the ROM (depth).

Next, write the RTL that will increment the address supplied to the ROM every \textbf{1/25th of a second}. The data coming out of the ROM should be fed to the \verb|tone| output. The ROM's address input should go from 0 to the depth of the ROM and should then loop around back to 0. You don't have a reset signal, so define the initial state of any registers in your design for simulation purposes. Also hook up the \verb|rom_address| output to the ROM address currently being accessed.

Now that you have implemented \verb|music_streamer|, create an instance of it in the module \verb|z1top.v|. Use the instance name \verb|streamer| to match the expected name in the \verb|.do| file. Instantiate a \verb|tone_generator| and wire \verb|SWITCHES[1]| to \verb|output_enable|, \verb|CLK_125MHZ_FPGA| to \verb|clk|, and \verb|aud_pwm| to \verb|square_wave_out|.  Assign \verb|aud_sd| to 1.  Connect the \verb|tone| output of the \verb|music_streamer| to the \verb|tone_switch_period| input of the \verb|tone_generator|. Connect the \verb|music_streamer|'s \verb|clk| input to the global clock signal. Finally, connect  its \verb|rom_address| output to the \verb|LEDS|s by routing the top 6 bits of address.

\section{Simulating the music\_streamer}
To simulate your \verb|music_streamer| open up the \verb|lab3/src/music_streamer_testbench.v|. In contrast to the \verb|tone_generator_testbench| where the \verb|tone_generator| was instantiated in isolation, in this testbench we are instantiating our entire top-level design, \verb|z1top|. This testbench is referred to as a system-level testbench, which tests our entire design using top-level I/O, in contrast to the \verb|tone_generator_testbench| which is a block-level testbench. This is similar to the difference between unit and integration tests in software development.

You can see that this testbench just runs a simulation for 2 seconds and then exits. You might have to modify the \verb|music_streamer_testbench.do| file to match the name of your module instances in \verb|z1top.v|.

To execute the testbench, run \verb|make CASES=tests/music_streamer_testbench.do| in \verb|lab3/sim|. This may take several minutes to complete. You may have to run \verb|make clean| before running \verb|make| if ModelSim has cached build artifacts.

Inspect your waveform to make sure you get what you expect. Verify that there are no undefined signals (red lines, x) Then run the Python script to generate a \verb|.wav| file of your simulation results and listen to your \verb|music_streamer|. It should sound like the first few seconds of the song that was loaded on the ROM.

\section{Verify your Code Works For Rest Notes}
In simulation, you can often catch bugs that would be difficult or impossible to catch by running your circuit on the FPGA. You should verify that if your ROM contains an entry that is zero (i.e. generate a 0Hz wave), that the \verb|tone_generator| holds the \verb|square_wave_out| output at either 1 or 0 with no oscillation. Verify this in simulation, and prove the correct functionality during checkoff.

\section{Try it on the FPGA!}
Now try your \verb|music_streamer| on the FPGA. You should expect the output to be the same as in simulation. The \verb|SWITCHES[1]| switch should still work to disable the output of the \verb|tone_generator|. Show your final results, simulation, and the working design on the FPGA to the TA for checkoff.

\section{Checkoff}
\begin{enumerate}
  \item Prove that if the ROM contains an entry for a \verb|tone_switch_period| of 0, that the square wave doesn't oscillate.
  \item Show the working \verb|music_streamer| on the FPGA.
\end{enumerate}

\section*{Ackowlegement}
This lab is the result of the work of many EECS151/251 GSIs over the years including:
\begin{itemize}
\item Sp12: James Parker, Daiwei Li, Shaoyi Cheng
\item Sp13: Shaoyi Cheng, Vincent Lee
\item Fa14: Simon Scott, Ian Juch
\item Fa15: James Martin
\item Fa16: Vighnesh Iyer
\item Fa17: George Alexandrov, Vighnesh Iyer, Nathan Narevsky
\item Sp18: Arya Reais-Parsi, Taehwan Kim
\item Fa18: Ali Moin, George Alexandrov, Andy Zhou
\end{itemize}

\end{document}
